\section{Appens opbygning}

\subsection{Overordnede opbygning}
Appen er bygget op omkring en enkelt aktivitet med mange tilhørende fragmenter. Dette har vi hovedsageligt valgt, da vores side-menu skal være synlig på alle sider af appen. Dermed kan man lade aktiviteten have side-menuen og en fragment-beholder, som den deri kan udfylde med forskellige fragmenter. Udover det, giver det også den fordel, at man ikke behøver at dele data på tværs af forskellige aktiviteter --- man kan bruge delte viewmodels til at dele data mellem fragmenter frem for forskellige Singleton Data Holders, som også ville være tilgængelige for andre elementer i vores Application Scope.

Navigation mellem fragmenter og dialog-bokse foregår med en navigationsgraf i xml, der definerer alle actions og parametre.

\subsection{Datagrundlag}
Appen indeholder data om forskellige planter, som er taget fra Årstidernes plantekalender \hyperlink{https://issuu.com/aarstiderne\_com/docs/aarstidernes-plante-og-saa-kalender}{https://issuu.com/aarstiderne\_com/docs/aarstidernes-plante-og-saa-kalender}.

Dataet er lagt ind i en \texttt{.csv}-fil, som læses ind i \texttt{PlantViewModel} via en \texttt{CSVReader}. På den måde er det nemt, hvis man skulle få mere data, eller at dataet ændrer sig. 

\subsection{Datarepræsentation}

Vi har syv datastrukturer, der abstraherer bede, planter, sæsoner mm. Herunder har vi både en \texttt{Plant} og en \texttt{MyPlant}, så vi kan skelne mellem en generelt plantetype med information om f.eks. sædskifteinterval og en plante, der er plantet i et bed og dermed har information om f.eks. forspiringsstatus knyttet til sig. Fordelen ved at skille dem ad på denne måde er, at man ikke behøver gemme generel info for hver plante af samme type, der er placeret i et bed.

\subsection{User interface}
For at bygge UI har vi brugt XML-filer. Her har vi hovedsageligt benyttet \texttt{ConstraintLayout}, da det er Googles standard layout. Ved at bruge dette layout frem for andre kan man også undgå lange og dybe hierarkier at views, der ligger inde i hinanden, da alt har en eller flere restriktioner frem for en tilhørende gruppe.

Derudover har vi besluttet, at appen kun skal bruges i portræt. Hvis man lægger sin telefonskærm ned, sker der ikke noget. Det har vi vurderet giver den bedste brugeroplevelse, da man f.eks. kan placere bed i forhold til hinanden, og ikke ønsker at de skal ændre placering ved en rotation af skærmen. 

\subsection{Prædikater}
Appen indeholder mange klasser, der arver fra vores eget interface \texttt{Predicate}. Dette er der to store grunde til:

\begin{enumerate}
    \item Mange steder i koden skal agere foreskelligt baseret på flere faktorer. Det ville kræve mange if-statements at tjekke alt igennem. Dette kan løses ved at sende et prædikat med. Kode-delen skal dermed bare tjekke om en boolean er opfyldt, uanset hvad, der kræves tjekket.
    \item Navigationsgrafen har begrænsninger på, hvad man kan sende med. Der er bestemte primitive typer, du kan bruge som parametre --- alternativt skal du oprette en klasse. Det er dermed ikke muligt f.eks. at bruge en parameter af typen \texttt{Bed -> Boolean}. Dette kan løses ved at lave et interface af typen \texttt{(T) -> Boolean}, som hver enkelt implementation kan specificere.
    
\end{enumerate}

\subsection{Lokation og vejrdata}
For at kunne vise vejrdata fra brugerens egen lokation, har vi lavet en Service, der begynder, når appen startes, og stopper, når den har en lokation. Ved at bruge en Service frem for asynkron kode, undgår vi race conditions, da datatet bliver udveklset igennem broadcasts og derfor ikke bliver eksekveret før en broadcast bliver modtaget.

Denne lokation bruger vi til at definere en boundary box, som gives videre til DMIs API for at få vejrdata fra den nærmeste vejrstation. Vi bruger DMI, da vores app er udviklet til brug i Danmark. 

\subsection{Lagring af data}
Vi gør brug af en lokal Room-database til at gemme de bede, sæsoner og egne planter brugeren opretter. Vi bruger en Room-database, da det er en del af Androidx og officielt understøttet af Google / Android. 

Til at tilgå vores data gør vi brug af DAO'er - Data Access Objects. Disse er en abstraktion til at tilgå, opdatere, slette mv. i databasen på en ensartet måde, så vi ikke skal skrive queries direkte til databasen hver gang vi ønsker at bruge eller ændre data deri.
For at gøre arbejdet med disse DAO'er lidt lettere for os har vi lavet repositories, som laver en yderligere abstraktion, for at vi kan kalde metoderne (såsom \texttt{updateBed}, \texttt{insertBed} osv.) asynkront, så vi ikke blokerer for UI- / Main-tråden mens vi lægger data i databasen.

Selve databasen er en singleton-instans, så vi ikke behøver at lave en ny instans hver gang vi skal tilgå databasen, da det er en relativt omstændig process. Dette er også anbefalet af Google på deres Android developer side \hyperlink{https://developer.android.com/training/data-storage/room/#database}{https://developer.android.com/training/data-storage/room/#database}
