\section{Refleksion}

\subsection{Indledende forventninger}
Da vi startede projektet, skrev vi følgende problemformulering:
\medskip


Derudover havde vi internt i gruppen nogle forestillinger om appens fremtidige kunnen:

\begin{enumerate}
    \item Guide dig fuldstændig i alt, hvad du foretager dig i din køkkenhave. Herunder indgår vanding, plantning/såning, forspiring og udplantning, sædskifte og høst
    \item Oversigter over alle handlinger, der skal udføres
    \item Understøtte planter indenfor, udenfor og i drivhus
    \item Understøtte at plante/så direkte i jorden, i potter eller i plantekasser
    \item Designe sit eget haveområde i appen
    \item Se historik over vanding og platning
    \item Planlægning af en ny sæson
\end{enumerate}

\subsection{Appens egentlige rolle}

\subsection{Appens omfang}
Her noget om hvilke forvetninger/krav vi opfyldte, og hvilke vi ikke gjorde.

\subsubsection{Kalenderintegration og notifikationer}
Da appens egentlige rolle stod mere klart, gik det også op for os, at kalenderintegration ikke gav meget mening at lave. Da appen ikke skulle holde styr på præcist hvilke handlinger, der skulle udføres hvornår, i hvilke bede og med hvilke planter, er der ikke nogen begivenheder, der kan stå i kalenderen. Hvis kalenderen skulle bruges, skulle der være afsat flere dage i træk til f.eks. at vise, at en plante var i sæson. Det vurderede vi ikke gav værdi for brugeren, da kalenderen som regel bruges til at holde styr på begivenheder eller arbejdsopgaver, man skal på et bestemt tidspunkt. Det er ikke en bogholder for hvornår du måske vil gøre noget.

Notifikationer gav stadig mening til en hvis grad, da de ville kunne minde brugeren om at tjekke til deres køkkenhave i ny og næ, ved ønske om sådanne påmindelser. Udfordringen hertil kommer af, at der er dårlig offline support for baggrundsservices på nyere versioner af Android, da de kan dræne ressourcer hele tiden. Dette gav os to muligheder for at få notifikationer ud til brugerne:

\begin{enumerate}
    \item Lav en uelegant løsning ved at bruge en \texttt{AlarmManager}, som fungerer som "alarm" og checker en gang i døgnet, om der skal sendes en notifikation til brugeren
    \item Lav en stor opsætning med Firebase eller en anden online service, som vi herfra kan få appen til at være "subscribet" til.
\end{enumerate}

Den første løsning ønskede vi ikke at lave, da det ville give en dårlig kodestruktur og misbruge \texttt{AlarmManager}-klassen, og gå imod den tiltænkte brug.

Den anden løsning blev fravalgt, da der skulle meget arbejde til at få server og service sat op, i forhold til hvilken værdi vi vurderede, det ville give for brugeren af appen. Det er overdrevent at gøre appen afhængig af en seperat server med en internet-service af den ene grund, at brugeren skal kunne få notifikationer om sin egen køkkenhave, som ellers ikke kræver internet at bruge.

\subsubsection{Sædskifteinterval}
Jævnfør problemformuleringen opfyldte vi dette krav. I forhold til vores egne forvetninger, derimod, blev det implemeteret på en helt anden måde. Det datagrundlag, vi fandt, samt det, vi læste om sædskifte (INDSÆT REFERENCE HER) informerede os om vigtighigheden heraf og hvordan det er forskelligt fra plante til plante. Derfor havde vi først troet, at vi skulle visualisere alle planter i alle bed i forhold til sædskifte. Efter en observation af bruger 1's nuværende måde at holde styr på sædskifte samt en forklarende snak med hende, fandt vi dog frem til, at det giver bedre mening at holde styr på sædskifte på bed-niveau. Man vil som oftest så planter af samme familie i samme bed og rotere hele bede for at undgå udpining af jorden. Derfor ændrede vi vores måde at anskue hele featuren til at give et overblik over hvor længe et bed har været på samme plads, hvor det tidligere har været samt hvornår det igen kan plantes på en tidligere position ifølge den type plante i bedet, der har største sædskifteinterval.

\subsubsection{Image analysis}
Vi havde snakket om, at implementere brug af image analysis til at hjælpe brugeren med at oprette sin have i appen ved blot at tage et billede af den. Dette vurderede vi ikke ville give værdi for brugeren da:
\begin{enumerate}
        \item Featuren bliver sandsynligvis kun brugt første gang, appen bruges, da man som udgangspunkt ikke ønsker at bygge sin have på ny efterfølgende. 
        \item Det kræver at dine planter er vokset nok til at de kan genkendes af en algoritme
\end{enumerate}

Derefter fik vi idéen fra en bruger, at vi kunne give mulighed for at tage et billede af en frøpose, og vi tænkte i den sammenhæng, at man kunne bruge image analysis til en billede-til-tekst funktion i notesfeltet. Konklusionen hertil blev, at det var for tidskrævende at integrere image analysis for en sådan lille forbedring af brugeroplevelsen, og det er derfor blevet til, at man må udfylde notesfeltet manuelt. 

Image analysis blev derfor ikke relevant for dette projekt.

\subsubsection{IoT devices}
Undervejs i projektet undersøgte vi hvilke køkkenhave-relaterede produkter, der enten allerede fandtes som IoT eller kunne give mening at gøre til en IoT.

\subsubsection*{Jordtermometer}
Der findes mange jordtermometre på markedet, men det er ikke lykkedes os at finde nogen IoT-jordtermometre. Derudover lagde vi mærke til, at de heller ikke kommer med en guide til hvilke planter, der tåler hvilken temperatur. Dette skal brugeren selv finde ud af ved f.eks. at købe frøposer med denne information på forsiden, søge på internettet eller læse i bøger. 

Det kunne derfor måske give mening at lave et samarbejde mellem appen og et jordtermometer. Det ville kræve, at man først lavede en række brugertests med brugere af jordtermometre, for at opleve deres arbejdspraksis og se, om det giver mening for dem at bruge en app.

I det tilfælde, at brugerne oplever det vanskeligt at finde information, at de generelt har meget at holde styr eller på anden vis kan få værdi af en app, ville det være oplagt at lave. Man kunne udvide appens sidemenu med en "Jordtemperatur"-tab. Herinde kunne man måske bruge noget bluetooth eller lignende til at forbinde til sit jordtermometer og appen ville derefter kunne vise dig information såsom:

\begin{enumerate}
    \item{Jordtemperaturen}
    \item{Planter, der kan klare denne jordtemperatur}
    \item{Hvordan de planter, du allerede har sået, tåler denne jordtemperatur}
\end{enumerate}

Derudover kunne man udvide leksikonnet til at have information om planternes fortrukne jodtemperatur.

For at denne idé kunne realiseres, ville man behøve et samarbejde med en udgiver af jordtermometre, samt yderligere data om planter i forhold til jordtemperatur.

\subsubsection*{Gardena vandingssystem}
Gardena Smart System har en app, der kan forbindes til et vandingssystem og/eller græsslåmaskine. Her er vandingssystemet relevant for køkkenhaver og drivhuse. Deres produkt har en sensor, der afgør hvornår der skal vandes. Den kigger også på vejrudsigten. Brugeren altid sætte den igang eller stoppe en vanding, samt sætte en tidsplan.

Dette produkt er klart mere nøjagtig og automatisk end vores feature, der kun tager højde for vejret og ikke er tilknyttet et produkt.

Man kunne måske lave et samarbejde med Gardena og få andre features med i deres app, så man på den måde også kan holde styr på sin køkkenhave. Der er dog flere pointer, der taler i mod sådan et samarbejde:

\begin{enumerate}
    \item{Gardena lægger på deres hjemmeside meget vægt på, at systemet klarer alt for dig. De skriver b.la. at man kan tage på ferie uden at bekymre sig (REFERENCE TIL SIDEN HER). Vores app lægger mere vægt på udbredning af information, der gør brugeren i stand til selv at handle og tage beslutninger. For at opklare dette punkt, ville man blive nødt til at tale med både brugere af Gardena og (fremtidige) brugere af vores app, samt Gardena for at afstemme forventninger, behov og mål for appen.}
    \item{Appen ville kunne alt for meget og dermed blive forvirrende. Her ville det igen være nødvendigt at interview eller observere adskillige bruger for at klarlægge dette.}
    \item{Gardena lader til gerne at ville omfavne haven som helhed, i og med, at de også har en græsslåmaskine, og ikke blot køkkenhavedelen. De ville måske ikke have lyst til at fokusere så meget på køkkenhave, som de features vi har implementeret. Dette ville kræve møder med Gardena for at klarlægge deres forretningsstrategi}
\end{enumerate}

Hvis disse pointer alle blev taget i betragtning, klarlagt og pegede på, at et samarbejde ville give god værdi for brugerne og virksomheden, ville en samlægning af de to apps virkelig kunne give styrke - man ville få en app, der gav brugeren mulighed for frihed eller kontrol efter ønske samt give dem meget information og mange features. 



