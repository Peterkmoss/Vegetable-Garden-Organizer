\section{Refleksion}
\label{refleksion}

\subsection{Indledende forventninger}
Jævnfør problemformulering fra indledningen, havde vi følgende mål (engelsk):

\begin{itemize}
        \item
    The ability to see what you have planted and where
        \item
    An overview of the actions you need to perform and when
        \item
    An overview of crop rotation for your vegetable garden
        \item
    Notifications about your crops and integration with calendar
        \item
    Overview of your plating locations (outdoors, greenhouse, etc.)
        \item
    Lexicon with additional info about plants 
        \item
    Research: integration with sensors or other IOT devices
        \item
    Research: Image analysis for identifying crops
\end{itemize}

Derudover havde vi internt i gruppen nogle forestillinger om appens fremtidige kunnen:

\begin{itemize}
    \item Guide dig fuldstændig i alt, hvad du foretager dig i din køkkenhave. Herunder indgår vanding, plantning/såning, forspiring og udplantning, sædskifte og høst
    \item Oversigter over alle handlinger, der skal udføres
    \item Understøtte planter indenfor, udenfor og i drivhus
    \item Understøtte at plante/så direkte i jorden, i potter eller i plantekasser
    \item Designe sit eget haveområde i appen
    \item Se historik over vanding og plantning
    \item Planlægning af en ny sæson
\end{itemize}

Vi blev dog klogere undervejs og opdagede, at ikke alle disse punkter var gode idéer, samt nogle punkter var for tidskrævende.

\subsection{Appens egentlige rolle}
Efter projektet kom undervejs og vi fik lavet en brugertest på bruger 1, opdagede vi hurtigt at hun ikke ønskede en app, som gjorde alt for hende. Hendes ønske til appen var at hjælpe hende med at huske og planlægge hvor hendes planter er og hvordan hun skal passe dem. Det blev derfor hurtigt klart for os, at fokus skulle ændres fra at være en app, som sørgede for at guide dig fuldstændig i alt hvad du foretager dig i din køkkenhave, til at være en app, som hjælper dig med at planlægge og give overblik i din køkkenhave.

\subsection{Appens omfang}
Her noget om hvilke forventninger/krav vi opfyldte, og hvilke vi ikke gjorde.

\subsubsection{Kalenderintegration og notifikationer}
Da appens egentlige rolle stod mere klart, gik det også op for os, at kalenderintegration ikke gav meget mening at lave. Da appen ikke skulle holde styr på præcist hvilke handlinger, der skulle udføres hvornår, i hvilke bede og med hvilke planter, er der ikke nogen begivenheder, der kan stå i kalenderen. Hvis kalenderen skulle bruges, skulle der være afsat flere dage i træk til f.eks. at vise, at en plante var i sæson. Det vurderede vi ikke gav værdi for brugeren, da kalenderen som regel bruges til at holde styr på begivenheder eller arbejdsopgaver, man skal på et bestemt tidspunkt. Det er ikke en bogholder for hvornår du måske vil gøre noget.

Notifikationer gav stadig mening til en hvis grad, da de ville kunne minde brugeren om at tjekke til deres køkkenhave i ny og næ, ved ønske om sådanne påmindelser. Udfordringen hertil kommer af, at der er dårlig offline support for baggrundsservices på nyere versioner af Android, da de kan dræne ressourcer hele tiden. Dette gav os to muligheder for at få notifikationer ud til brugerne:

\begin{enumerate}
    \item Lav en uelegant løsning ved at bruge en \texttt{AlarmManager}, som fungerer som ``alarm'' og checker en gang i døgnet, om der skal sendes en notifikation til brugeren
    \item Lav en stor opsætning med Firebase eller en anden online service, som vi herfra kan få appen til at være ``subscribet'' til.
\end{enumerate}

Den første løsning ønskede vi ikke at lave, da det ville give en dårlig kodestruktur og misbruge \texttt{AlarmManager}-klassen, og gå imod den tiltænkte brug.

Den anden løsning blev fravalgt, da der skulle meget arbejde til at få server og service sat op, i forhold til hvilken værdi vi vurderede, det ville give for brugeren af appen. Det er overdrevent at gøre appen afhængig af en separat server med en internet-service af den ene grund, at brugeren skal kunne få notifikationer om sin egen køkkenhave, som ellers ikke kræver internet at bruge.

\subsubsection{Sædskifteinterval}
Jævnfør problemformuleringen opfyldte vi dette krav. I forhold til vores egne forventninger, derimod, blev det implementeret på en helt anden måde. Det datagrundlag, vi fandt, samt det, vi læste om sædskifte\footcite{sædskifte} informerede os om vigtigheden heraf og hvordan det er forskelligt fra plante til plante. Derfor havde vi først troet, at vi skulle visualisere alle planter i alle bed i forhold til sædskifte. Efter en observation af bruger 1's nuværende måde at holde styr på sædskifte samt en forklarende snak med hende, fandt vi dog frem til, at det giver bedre mening at holde styr på sædskifte på bed-niveau. Man vil som oftest så planter af samme familie i samme bed og rotere hele bede for at undgå udpining af jorden. Derfor ændrede vi vores måde at anskue hele featuren til at give et overblik over hvor længe et bed har været på samme plads, hvor det tidligere har været samt hvornår det igen kan plantes på en tidligere position ifølge den type plante i bedet, der har største sædskifteinterval.

\subsubsection{Image analysis}
\label{image-analysis}
Vi havde snakket om, at implementere brug af image analysis til at hjælpe brugeren med at oprette sin have i appen ved blot at tage et billede af den. Dette vurderede vi ikke ville give værdi for brugeren da:
\begin{enumerate}
        \item Featuren bliver sandsynligvis kun brugt første gang, appen bruges, da man som udgangspunkt ikke ønsker at bygge sin have på ny efterfølgende. 
        \item Det kræver at dine planter er vokset nok til at de kan genkendes af en algoritme
\end{enumerate}

Derefter fik vi idéen fra en bruger, at vi kunne give mulighed for at tage et billede af en frøpose, og vi tænkte i den sammenhæng, at man kunne bruge image analysis til en billede-til-tekst funktion i notefeltet. Konklusionen hertil blev, at det var for tidskrævende at integrere image analysis for en sådan lille forbedring af brugeroplevelsen, og det er derfor blevet til, at man må udfylde notefeltet manuelt. 

Image analysis blev derfor ikke relevant for dette projekt.

\subsubsection{IoT devices}
I slutningen af projektet lavede vi et interview med Sebastian Büttrich, der er Research Lab Manager på ITU samt underviser i et IoT-kursus, også på ITU. Han har lavet meget hvad der angår IoT enheder, som kan måle forskellige forhold, en plante har. Vi lavede dette interview for at supplere vores viden om hvad, der findes af IoT enheder på markedet og deres rolle i forhold til forbrugerne, samt generelt at få et større indblik omkring området. Listen over de produkter, vi har undersøgt kan ses i bilag \ref{products-on-market}.

\subsubsection*{Fugtighed, lys, temperatur og pH}
Der findes mange enheder til at måle jordtemperatur og -fugtighed, lys og pH, som kan hjælpe dig med at afgøre hvornår, du skal udføre en bestemt handling. Heraf findes der både produkter som IoT, almindelige analog måleinstrumenter samt færdige løsninger, der kan udføre handlingen for dig.

\subsubsection*{Kamera og machine learning}
Én ting som Sebastian nævnte, at han synes kunne være en god ting at få beskeder om i forhold til sine planter, var i forhold til skadedyr. Her kunne han godt se nytten i at få information om hvorvidt ens plante var ramt af f.eks. bladlus eller andre skadedyr eller sygdomme. Han nævnte at der lige nu findes små kamera-moduler, som bruger image analysis og ML til at fortælle om en plante mangler f.eks. kalk, grundet misfarvning i bladene. Dette er meget plante-specifikt og derudover meget ressourcekrævende, da der skal være et kamera på hver plante, men dog er det muligt.

\subsubsection*{Vurdering}
Mange a produkterne beskrevet i de to underafsnit foroven, ville i fremtiden kunne blive integreret med vores app. Her kunne man forestille sig, at brugeren i forbindelse med appen, kan få adgang til data om jordtemperaturen samt få en oversigt over hvad, de kunne plante under disse omstændigheder. Dette ville kræve en række brugertests for at afdække, om denne funktionalitet giver værdi.

Sebastian fortalte, at mange af de mere præcise og avancerede IoT-enheder ofte kun bliver brugt til professionelt landbrug, da de er for dyre og specialicerede at bruge hjemme i en køkkenhave. Priserne er dog på vej ned, men der går alligevel et stykke tid, før det giver mening for en hobbyist at anskaffe sig IoT måleinstrumenter. De fleste produkter til en betalelig pris på markedet lige nu (se bilag \ref{products-on-market}), vil være upræcise, og det vil derfor i langt de fleste tilfælde være langt mere sigende blot at kigge til planterne selv. F.eks. kan placeringen af en sensor i jorden have stor betydning for resultatet af målingen.

Sebastian kom derudover med en pointe om, at automatiserede systemer kan gå imod idéen med at have en køkkenhave. Han tror at de mennesker, som holder sin egen køkkenhave, inklusiv ham selv, har det som en hobby --- hvis man derfor begynder at automatisere hele processen med at have en køkkenhave, kan meget af glæden gå af at passe sin have og sine planter, da hobbyen næsten ikke kommer til udtryk i dit liv. Dette udspurgte vi vores to brugere omkring. Her svarede bruger 1, at hun kun var delvis enig og godt kunne tænkte sig automatisering til en vis grad. Det tyder på at det kan være meget forskelligt, og det ville derfor være nødvendigt at lave mange brugertests for at afdække behovet og ønsket.

Vi vurderede derfor, at hvis vi i fremtiden ønsker at inkludere data fra IoT-enheder, vil det sandsynligvis give bedst mening at have til formål at vise brugeren hvad, der foregår i deres køkkenhave, frem for præcis hvilken handling, de skal udføre eller en fuldstændig automatisering.
