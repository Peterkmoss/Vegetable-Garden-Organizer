\section{Fremtidige udvidelser}
Gennem projektet nåede vi alt i backloggen, som det også fremgik i vores burn-up chart. Der var dog enkelte ting, vi fjernede fra backloggen undervejs, da det ikke længere var del af vores ønskede mål og omfang for arbejdet frem mod afleveringen --- nogle af disse features kunne dog måske stadig give mening at implementere i et fremtidigt arbejde på appen. Derudover fik vi også nye idéer undervejs, der eventuelt kan give værdi for brugeren i fremtiden.

\subsection{Ikoner til alle planter}
\label{icons-for-plants}
Nogle mennesker orienterer sig meget visuelt, og her ville det give god mening, hvis hver enkelt plante havde et ikon knyttet til sig. Da dette kobler sig meget til datagrundlaget, vurderede vi, at det ikke gav mening at bruge tid på at finde hvert enkelt ikon til hver af de planter, vi havde fra vores data. Det ville være oplagt at have data, der allerede havde knyttet et ikon til sig. Alternativt ville det være meget tidskrævende, at finde ikoner samt skalere og farve dem. Det vurderede vi ikke gav mening i forhold til vores nuværende tidsramme, især når der heller ikke var nogen brugertests, der fremhævede manglen. I en fremtidig version af appen, ville det måske give mening at få et samarbejde, der kunne levere alt data herunder ikoner. 

\subsection{Inddeling og gruppering af bede}
\label{gruppering-af-bede}
Baseret på vores brugertests beskrevet tidligere, var der ønske om en større inddeling af bede. Det ville derfor være oplagt at implementere i en fremtidig version. Dette ønske vurderede vi ikke, der var tid til at implementere på en optimal måde, da ønskerne først kom efter sprint 7 ud af 12. Hvis man kigger på vores burn-up chart (se figur \ref{burn-up}), lader det til at være en god beslutning, da vi først nåede i mål med vores backlog i sidste sprint, kun med få story points mindre end andre sprints.

\subsection{Sortering af listen over planter}
Når brugeren skal oprette eller redigere et bed, får de en liste af alle prædefinerede og selvoprettede planter. Der er lige nu ingen sortering i den rækkefølge, de bliver vist i.  Man kunne lave en lignende sortering i listen over planter, der er i sæson. Her er der flere forskellige muligheder:
\begin{itemize}
   \item Lade brugeren definere favorit planter, som vises først
   \item Sortere på senest brugt. Dette er især nyttigt, hvis brugeren skal indtaste samme plante flere gange i træk
   \item Alfabetisk rækkefølge
   \item Brugerdefinerede planter øverst
\end{itemize}

Uanset hvilken metode, man valgte, ville det give værdi for brugeren, da der lige nu ikke er noget system, så man kan ikke regne sig frem til hvor i listen, en plante findes.

\subsection{Tilpasset layout i oversigt over bede}
Som det ser ud lige nu, er der altid maksimum tre kolonner, når man kan se alle sine bede enten udendørs eller i drivhus. Det kunne måske skabe irritation for nogle brugere, hvis de f.eks.\ har fire bede ved siden af hinanden geografisk i deres have og dermed bliver nødt til at huske, at appens repræsentation ikke matcher positionen i virkeligheden. 

For at opnå den ønskede funktionalitet kunne man lave en redigér-knap lignende den, man finder inde på et enkelt bed, der giver muglihed for at specificere hvor mange rækker og kolonner man ønsker at se oversigten i.

\subsection{Flet celler}
Hvis man har planter, der fylder mere i bredden end andre, eller som skal sås med større afstand, ville det være visuelt nemmere at holde styr på, hvis der var mulighed for at flette celler. Det ville give mulighed for at repræsentere haven endnu mere detaljeret end lige nu.

\subsection{To-do liste}
Appen har en funktion, der giver mulighed for at skrive noter for den enkelte plante. Den giver derimod ikke mulighed for at skrive generelle noter eller lave en overordnet to-do liste. Her kunne man lave en ny tab i side-menuen, der gav adgang til at lave en to-do liste og skrive noter.
