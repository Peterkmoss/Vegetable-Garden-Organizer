\label{refleksion}
\section{Refleksion}

\subsection{Indledende forventninger}
Jævnfør problemformulering fra indledningen, havde vi følgende mål:

\begin{itemize}
        \item
    The ability to see what you have planted and where
        \item
    An overview of the actions you need to perform and when
        \item
    An overview of crop rotation for your vegetable garden
        \item
    Notifications about your crops and integration with calendar
        \item
    Overview of your plating locations (outdoors, greenhouse, etc.)
        \item
    Lexicon with additional info about plants 
        \item
    Research: integration with sensors or other IOT devices
        \item
    Research: Image analysis for identifying crops
\end{itemize}

Derudover havde vi internt i gruppen nogle forestillinger om appens fremtidige kunnen:

\begin{itemize}
    \item Guide dig fuldstændig i alt, hvad du foretager dig i din køkkenhave. Herunder indgår vanding, plantning/såning, forspiring og udplantning, sædskifte og høst
    \item Oversigter over alle handlinger, der skal udføres
    \item Understøtte planter indenfor, udenfor og i drivhus
    \item Understøtte at plante/så direkte i jorden, i potter eller i plantekasser
    \item Designe sit eget haveområde i appen
    \item Se historik over vanding og platning
    \item Planlægning af en ny sæson
\end{itemize}

Vi blev dog klogere undervejs og opdagede, at ikke alle disse punkter var gode idéer, samt nogle punkter var for tidskrævende.

\subsection{Appens egentlige rolle}
Her noget om skiftet af fokus fra at appen kan alt til at den viser alt.

\subsection{Appens omfang}
Her noget om hvilke forvetninger/krav vi opfyldte, og hvilke vi ikke gjorde.

\subsubsection{Kalenderintegration og notifikationer}
Da appens egentlige rolle stod mere klart, gik det også op for os, at kalenderintegration ikke gav meget mening at lave. Da appen ikke skulle holde styr på præcist hvilke handlinger, der skulle udføres hvornår, i hvilke bede og med hvilke planter, er der ikke nogen begivenheder, der kan stå i kalenderen. Hvis kalenderen skulle bruges, skulle der være afsat flere dage i træk til f.eks. at vise, at en plante var i sæson. Det vurderede vi ikke gav værdi for brugeren, da kalenderen som regel bruges til at holde styr på begivenheder eller arbejdsopgaver, man skal på et bestemt tidspunkt. Det er ikke en bogholder for hvornår du måske vil gøre noget.

Notifikationer gav stadig mening til en hvis grad, da de ville kunne minde brugeren om at tjekke til deres køkkenhave i ny og næ, ved ønske om sådanne påmindelser. Udfordringen hertil kommer af, at der er dårlig offline support for baggrundsservices på nyere versioner af Android, da de kan dræne ressourcer hele tiden. Dette gav os to muligheder for at få notifikationer ud til brugerne:

\begin{enumerate}
    \item Lav en uelegant løsning ved at bruge en \texttt{AlarmManager}, som fungerer som "alarm" og checker en gang i døgnet, om der skal sendes en notifikation til brugeren
    \item Lav en stor opsætning med Firebase eller en anden online service, som vi herfra kan få appen til at være "subscribet" til.
\end{enumerate}

Den første løsning ønskede vi ikke at lave, da det ville give en dårlig kodestruktur og misbruge \texttt{AlarmManager}-klassen, og gå imod den tiltænkte brug.

Den anden løsning blev fravalgt, da der skulle meget arbejde til at få server og service sat op, i forhold til hvilken værdi vi vurderede, det ville give for brugeren af appen. Det er overdrevent at gøre appen afhængig af en seperat server med en internet-service af den ene grund, at brugeren skal kunne få notifikationer om sin egen køkkenhave, som ellers ikke kræver internet at bruge.

\subsubsection{Sædskifteinterval}
Jævnfør problemformuleringen opfyldte vi dette krav. I forhold til vores egne forvetninger, derimod, blev det implemeteret på en helt anden måde. Det datagrundlag, vi fandt, samt det, vi læste om sædskifte (INDSÆT REFERENCE HER) informerede os om vigtighigheden heraf og hvordan det er forskelligt fra plante til plante. Derfor havde vi først troet, at vi skulle visualisere alle planter i alle bed i forhold til sædskifte. Efter en observation af bruger 1's nuværende måde at holde styr på sædskifte samt en forklarende snak med hende, fandt vi dog frem til, at det giver bedre mening at holde styr på sædskifte på bed-niveau. Man vil som oftest så planter af samme familie i samme bed og rotere hele bede for at undgå udpining af jorden. Derfor ændrede vi vores måde at anskue hele featuren til at give et overblik over hvor længe et bed har været på samme plads, hvor det tidligere har været samt hvornår det igen kan plantes på en tidligere position ifølge den type plante i bedet, der har største sædskifteinterval.

\label{image-analysis}
\subsubsection{Image analysis}
Vi havde snakket om, at implementere brug af image analysis til at hjælpe brugeren med at oprette sin have i appen ved blot at tage et billede af den. Dette vurderede vi ikke ville give værdi for brugeren da:
\begin{enumerate}
        \item Featuren bliver sandsynligvis kun brugt første gang, appen bruges, da man som udgangspunkt ikke ønsker at bygge sin have på ny efterfølgende. 
        \item Det kræver at dine planter er vokset nok til at de kan genkendes af en algoritme
\end{enumerate}

Derefter fik vi idéen fra en bruger, at vi kunne give mulighed for at tage et billede af en frøpose, og vi tænkte i den sammenhæng, at man kunne bruge image analysis til en billede-til-tekst funktion i notesfeltet. Konklusionen hertil blev, at det var for tidskrævende at integrere image analysis for en sådan lille forbedring af brugeroplevelsen, og det er derfor blevet til, at man må udfylde notesfeltet manuelt. 

Image analysis blev derfor ikke relevant for dette projekt.

\subsubsection{IoT devices}
Undervejs i projektet undersøgte vi hvilke køkkenhave-relaterede produkter, der enten allerede fandtes som IoT eller kunne give mening at gøre til en IoT.

\subsubsection*{Jordtermometer}
Der findes mange jordtermometre på markedet, men det er ikke lykkedes os at finde nogen IoT-jordtermometre. Derudover lagde vi mærke til, at de heller ikke kommer med en guide til hvilke planter, der tåler hvilken temperatur. Dette skal brugeren selv finde ud af ved f.eks. at købe frøposer med denne information på forsiden, søge på internettet eller læse i bøger. 

Det kunne derfor måske give mening at lave et samarbejde mellem appen og et jordtermometer. Det ville kræve, at man først lavede en række brugertests med brugere af jordtermometre, for at opleve deres arbejdspraksis og se, om det giver mening for dem at bruge en app.

I det tilfælde, at brugerne oplever det vanskeligt at finde information, at de generelt har meget at holde styr eller på anden vis kan få værdi af en app, ville det være oplagt at lave. Man kunne udvide appens sidemenu med en "Jordtemperatur"-tab. Herinde kunne man måske bruge noget bluetooth eller lignende til at forbinde til sit jordtermometer og appen ville derefter kunne vise dig information såsom:

\begin{enumerate}
    \item{Jordtemperaturen}
    \item{Planter, der kan klare denne jordtemperatur}
    \item{Hvordan de planter, du allerede har sået, tåler denne jordtemperatur}
\end{enumerate}

Derudover kunne man udvide leksikonnet til at have information om planternes fortrukne jodtemperatur.

For at denne idé kunne realiseres, ville man behøve et samarbejde med en udgiver af jordtermometre, samt yderligere data om planter i forhold til jordtemperatur.

\subsubsection*{Møde med Sebastian}
Arbejder ikke med mobile apps
Hobby gardner, dyrker sjældne planter

Identifikation af planter og kan identificere, crowd sourcing
ITU - projekt, jordfugtighed, temperatur og lys - stick som man satte i jorden.
	Erregation system - automatisk vanding
	Social media beskeder - sendte beskeder

Fugtighedsmåler, lys og temperatur er meget normal, sparkfun, seedstudio, audafruit. Banggood / Aliexpress
Sælges med frø til planter
PH er ikke så normal
Jordbærdyrkning, industriel precicion agriculture

Fugtighedsmålere er ikke særlig præcise. Afhængig af kontakten med jorden, kan svinge meget.
Kan ikke bruge 100kr ting, skal bruge 1000kr ting (professionelle)

Man bliver tit nødt til at kigge med sine øjne. Plantens aktivitet.
Kan automatiseres, men plante-specifikt

Plante-container, nørrebro

Sebastian kører meget på routine, ofte en hobby.

Irma har krydderi-køleskabe agtige ting? Automatiseret plante

Kamerasystemer og ML for at identificere hvordan planen har det. Analyserer plantens farve i bladene.
Sjældent hobby mest professionelt

scholar.google.com, precision agriculture review / overview

Crowd sourcing, andre brugere indrapporterer data

Kamera-modul til arduino, portenta

Indgangsvinkel: hurtig udvikling, så forskningen ikke kan følge med (ift. IoT), de laver smart alting
Giver det noget at lave IoT til plante-dyrkning? Går det imod formålet med at have hobbyen?

Skadedyr - en app til at sige om de har skadedyr

Sebastian er Research Lab Manager på ITU, underviser IoT på ITU
https://www.tindie.com/products/miceuz/chirp-plant-watering-alarm/

https://www.sparkfun.com/products/18044

https://www.seeedstudio.com/Grove-Smart-Plant-Care-Kit-for-Arduino.html

https://www.adafruit.com/?q=plant&sort=BestMatch

https://www.banggood.com/6Pcs-or-12Pcs-Upgraded-Automatic-Watering-Device-Adjustable-Water-Flow-Dripper-With-Switch-Control-Valve-Bracket-Design-DIY-Drip-Irrigation-for-Plants-Indoor-Household-Waterers-Bottle-p-1527006.html?cur_warehouse=CN&ID=514326&rmmds=search

https://scholar.google.com/scholar?as_ylo=2021&q=precision+agriculture&hl=da&as_sdt=0,5

https://scholar.google.com/scholar?hl=da&as_sdt=0%2C5&as_ylo=2021&q=precision+agriculture+review&btnG=

Research Lab Manager https://DASYA.itu.dk underviser IoT

